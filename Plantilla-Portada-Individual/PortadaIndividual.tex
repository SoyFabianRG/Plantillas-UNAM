% ¡Hola! Esta plantilla fue diseñada por Fabián Ríos.
% Puedes encontrar el repositorio en: https://github.com/SoyFabianRG/Plantillas-UNAM-LaTeX

% =======================================================
% ==========             PREÁMBULO             ==========
% =======================================================

    % **********  ESPECIFICACIONES  ***************
    \documentclass[12pt,letterpaper]{article}                       % Tipo de documento y tamaño de fuente
    \usepackage[margin=2.54cm]{geometry}                            % Margen
    \usepackage{pdfpages}                                           % Permite "importar" archivos PDF

    % **********  IDIOMA Y CODIFICACIÓN  **********
    \usepackage[spanish]{babel}                                     % Idioma
    \usepackage[utf8]{inputenc}                                     % Codificación de entrada (caracteres especiales)
    \usepackage[T1]{fontenc}                                        % Codificación para un mejor soporte de fuentes
    \usepackage{lmodern}                                            % Fuente compatible: Latin Modern

    % **********  FIGURAS Y OBJETOS  **************
    \usepackage{graphicx}                                           % Permite colocar imágenes
    \usepackage[export]{adjustbox}                                  % Permite comparar imágenes
    \DeclareGraphicsExtensions{.pdf, .png, .jpg, .PNG, .JPG}        % Ya no es necesario especificar la extensión del archivo
    \graphicspath{ {../Graphics/} }                                 % Lugar donde están guardadas las imágenes

% =======================================================
% ==========              PORTADA              ==========
% =======================================================
\begin{document}
\begin{titlepage}
    \centering                                                      % Centra todo el texto

    \includegraphics[width=0.23\textwidth]                          % Si quieres los escudos en color negro, en el directorio "Escudos"
    {unam-logo-azul}                                                % los podrás encontrar. De otra manera, puedes agregar los escudos
    {\includegraphics[width=0.23\textwidth, right=12cm]             % de tu respectiva universidad y reemplazar la ruta del directorio
    {ciencias-logo-azul}\par}                                       % en el comando \includegraphics para utilizarlos

    \vspace{0.25cm}                                                 % Espacio entre párrafos
    {\bfseries\LARGE Universidad Nacional Autónoma de México \par}  % No hay mejor sensación que escribir esto en tus trabajos, ¿cierto?

    \vspace{0.25cm}                                                 % Espacio entre párrafos
    {\scshape\Large Facultad de Ciencias \par}                      % Nombre de la Facultad

    \vfill                                                          % Espacio entre párrafos
    {\bfseries\Huge TÍTULO \par}                                    % El título de tu tarea. Recuerda que debe estar en mayúsculas

    \vfill                                                          % Espacio entre párrafos
    {\bfseries\Large PRESENTA \par}
    \vspace{0.1cm}                                                  % En este apartado coloca tu nombre. Recuerda que debe estar en mayúsculas.
    {\large TU NOMBRE \\ TU NUMERO DE CUENTA \par}                  % Es opcional colocar tu número de cuenta, es importante cuidar tus datos

    \vfill                                                          % Espacio entre párrafos
    {\bfseries\Large PROFESOR/A \par}                               % Recuerda escribir PROFESORA o PROFESOR según sea el caso.
    \vspace{0.1cm}                                                  % ¡Usa un lenguaje inclusivo y con perspectiva de género!
    {\large NOMBRE DEL PROFESOR/A \par}                             % Nombre de tu profesor/a. Recuerda que debe estar en mayúsculas

    \vfill                                                          % Espacio entre párrafos
    {\bfseries\Large ASIGNATURA \par}
    \vspace{0.1cm}                                                  % Mini espacio
    {\large NOMBRE DE LA ASIGNATURA \par}                           % Nombre de la materia. Recuerda que debe estar en mayúsculas

    \vfill                                                          % Espacio entre párrafos
    {\Large GRUPO **** \par}                                        % Tu grupo de clase (reemplaza los asteriscos)

    \vfill                                                          % Espacio entre párrafos
    {\Large SEMESTRE **** \par}                                     % Semestre actual

    \vfill                                                          % Espacio entre párrafos
    {\Large \today \par}                                            % Fecha actual. También puedes agregar la fecha manualmente eliminando \today

\end{titlepage}
\end{document}